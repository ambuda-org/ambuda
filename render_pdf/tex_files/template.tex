
\documentclass{tufte-handout}

\title{śivopaniṣad}

\author{apauruṣeya}

%\date{28 March 2010} % without \date command, current date is supplied

%\geometry{showframe} % display margins for debugging page layout

\usepackage{graphicx} % allow embedded images
  \setkeys{Gin}{width=\linewidth,totalheight=\textheight,keepaspectratio}
  \graphicspath{{graphics/}} % set of paths to search for images
\usepackage{amsmath}  % extended mathematics
\usepackage{booktabs} % book-quality tables
\usepackage{units}    % non-stacked fractions and better unit spacing
\usepackage{multicol} % multiple column layout facilities
\usepackage{lipsum}   % filler text
\usepackage{fancyvrb} % extended verbatim environments
  \fvset{fontsize=\normalsize}% default font size for fancy-verbatim environments

% Standardize command font styles and environments
\newcommand{\doccmd}[1]{\texttt{\textbackslash#1}}% command name -- adds backslash automatically
\newcommand{\docopt}[1]{\ensuremath{\langle}\textrm{\textit{#1}}\ensuremath{\rangle}}% optional command argument
\newcommand{\docarg}[1]{\textrm{\textit{#1}}}% (required) command argument
\newcommand{\docenv}[1]{\textsf{#1}}% environment name
\newcommand{\docpkg}[1]{\texttt{#1}}% package name
\newcommand{\doccls}[1]{\texttt{#1}}% document class name
\newcommand{\docclsopt}[1]{\texttt{#1}}% document class option name
\newenvironment{docspec}{\begin{quote}\noindent}{\end{quote}}% command specification environment

\usepackage[parfill]{parskip}
\usepackage{fontspec}
\usepackage[english]{babel}
\usepackage{microtype}

% Sanskrit Support %
\babelprovide[import]{sanskrit-devanagari}
\babelprovide[import]{english}
\babelprovide[import]{greek}
\babelprovide[import]{german}
\babelfont[english]{rm}{Bitter}
\babelfont[sanskrit-devanagari]{rm}
          {Siddhanta}
\babelfont[sanskrit-devanagari]{sf}
          [Language=Default]{Noto Sans}

          
\newcommand\textsanskrit[1]{\foreignlanguage{sanskrit-devanagari}{#1}}
\newenvironment{sanskrit}%
{\begin{otherlanguage}{sanskrit-devanagari}}%
{\end{otherlanguage}}
%%%%%%%%%%%%%%%%%%


\begin{document}
\maketitle

 \ SU.1.1\\kailāsaśikharāsīnam aśeṣāmarapūjitam\\kālaghnaṃ śrīmahākālam īśvaraṃ jñānapāragam\\\  \ SU.1.2\\saṃpūjya vidhivad bhaktyā ṛṣyātreyaḥ susaṃyataḥ\\sarvabhūtahitārthāya papracchedaṃ mahāmuniḥ\\\  \ SU.1.3\\jñānayogaṃ na vindanti ye narā mandabuddhayaḥ\\te mucyante kathaṃ ghorād bhagavan bhavasāgarāt\\\  \ SU.1.4\\evaṃ pṛṣṭaḥ prasannātmā ṛṣy ātreyeṇa dhīmatā\\mandabuddhivimuktyarthaṃ mahākālaḥ prabhāṣate\\\  \ SU.1.5\\purā rudreṇa gaditāḥ śivadharmāḥ sanātanāḥ\\devyāḥ sarvagaṇānāṃ ca saṃkṣepād granthakoṭibhiḥ\\\  \ SU.1.6\\āyuḥ prajñāṃ tathā śaktiṃ prasamīkṣya nṝṇām iha\\tāpatrayaprapīḍāṃ ca bhogatṛṣṇāvimohinīm\\\  \ SU.1.7\\te dharmāḥ skandanandibhyām anyaiś ca munisattamaiḥ\\sāramādāya nirdiṣṭāḥ samyakprakaraṇāntaraiḥ\\\  \ SU.1.8\\sārād api mahāsāraṃ śivopaniṣadaṃ param\\alpagranthaṃ mahārthaṃ ca pravakṣyāmi jagaddhitam\\\  \ SU.1.9\\śivaḥ śiva ime śānta nāma cādyaṃ muhurmuhuḥ\\uccārayanti tad bhaktyā te śivā nātra saṃśayaḥ\\\  \ SU.1.10\\aśivāḥ pāśasaṃyuktāḥ paśavaḥ sarvacetanāḥ\\yasmād vilakṣaṇās tebhyas tasmād īśaḥ śivaḥ smṛtaḥ\\\  \ SU.1.11\\guṇo buddhir ahaṃkāras tanmātrāṇīndriyāni ca\\bhūtāni ca caturviṃśad iti pāśāḥ prakīrtitāḥ\\\  \ SU.1.12\\pañcaviṃśakam ajñānaṃ sahajaṃ sarvadehinām\\pāśājālasya tan mūlaṃ prakṛtiḥ kāraṇāya naḥ\\\  \ SU.1.13\\satyajñāne nibadhyante puruṣāḥ pāśabandhanaiḥ\\madbhāvāc ca vimucyante jñāninaḥ pāśapañjarāt\\\  \ SU.1.14\\ṣaḍviṃśakaś ca puruṣaḥ paśur ajñaḥ śivāgame\\saptaviṃśa iti proktaḥ śivaḥ sarvajagatpatiḥ\\\  \ SU.1.15\\yasmāc chivaḥ susaṃpūrṇaḥ sarvajñaḥ sarvagaḥ prabhuḥ\\tasmāt sa pāśaharitaḥ sa viśuddhaḥ svabhāvataḥ\\\  \ SU.1.16\\paśupāśaparaḥ śāntaḥ paramajñānadeśikaḥ\\śivaḥ śivāya bhūtānāṃ taṃ vijñāya vimucyate\\\  \ SU.1.17\\etad eva paraṃ jñānaṃ śiva ity akṣaradvayam\\vicārād yāti vistāraṃ tailabindur ivāmbhasi\\\  \ SU.1.18\\sakṛd uccāritaṃ yena śiva ity akṣaradvayam\\baddhaḥ parikaras tena mokṣopagamanaṃ prati\\\  \ SU.1.19\\dvyakṣaraḥ śivamantro 'yaṃ śivopaniṣadi smṛtaḥ\\ekākṣaraḥ punaś cāyam om ity evaṃ vyavasthitaḥ\\\  \ SU.1.20\\nāmasaṃkīrtaṇād eva śivasyāśeṣapātakaiḥ\\yataḥ pramucyate kṣipraṃ mantro 'yaṃ dvyakṣaraḥ paraḥ\\\  \ SU.1.21\\yaḥ śivaṃ śivam ity evaṃ dvyakṣaraṃ mantram abhyaset\\ekākṣaraṃ vā satataṃ sa yāti paramaṃ padam\\\  \ SU.1.22\\mitrasvajanabandhūnāṃ kuryān nāma śivātmakam\\api tat kīrtanād yāti pāpamuktaḥ śivaṃ puram\\\  \ SU.1.23\\vijñeyaḥ sa śivaḥ śānto naras tadbhāvabhāvitaḥ\\āste sadā nirudvignaḥ sa dehānte vimucyate\\\  \ SU.1.24\\hṛdy antaḥkaraṇaṃ jñeyaṃ śivasya āyatanaṃ param\\hṛtpadmaṃ vedikā tatra liṅgam oṃkāram iṣyate\\\  \ SU.1.25\\puruṣaḥ sthāpako jñeyaḥ satyaṃ saṃmārjanaṃ smṛtam\\ahiṃsā gomayaṃ proktaṃ śāntiś ca salilaṃ param\\\  \ SU.1.26\\kuryāt saṃmārjanaṃ prājño vairāgyaṃ candanaṃ smṛtam\\pūjayed dhyānayogena saṃtoṣaiḥ kusumaiḥ sitaiḥ\\\  \ SU.1.27\\dhūpaś ca guggulur deyaḥ prāṇāyāmasamudbhavaḥ\\pratyāhāraś ca naivedyam asteyaṃ ca pradakṣiṇam\\\  \ SU.1.28\\iti divyopacāraiś ca saṃpūjya paramaṃ śivam\\japed dhyāyec ca muktyarthaṃ sarvasaṅgavivarjitaḥ\\\  \ SU.1.29\\jñānayogavinirmuktaḥ karmayogasamāvṛttaḥ\\mṛtaḥ śivapuraṃ gacchet sa tena śivakarmaṇā\\\  \ SU.1.30\\tatra bhuktvā mahābhogān pralaye sarvadehinām\\śivadharmāc chivajñānaṃ prāpya muktim avāpnuyāt\\\  \ SU.1.31\\jñānayogena mucyante dehapātād anantaram\\bhogān bhuktvā ca mucyante pralaye karmayoginaḥ\\\  \ SU.1.32\\tasmāj jñānavido yogāt tathājñāḥ karmayoginaḥ\\sarva eva vimucyante ye narāḥ śivam āśritāḥ\\\  \ SU.1.33\\sa bhogaḥ śivavidyārthaṃ yeṣāṃ karmāsti nirmalam\\te bhogān prāpya mucyante pralaye śivavidyayā\\\  \ SU.1.34\\vidyā saṃkīrtanīyā hi yeṣāṃ karma na vidyate\\te cāvartya vimucyante yāvat karma na tad bhavet\\\  \ SU.1.35\\śivajñānavidaṃ tasmāt pūjayed vibhavair gurum\\vidyādānaṃ ca kurvīta bhogamokṣajigīṣayā\\\  \ SU.1.36\\śivayogī śivajñānī śivajāpī tapo'dhikaḥ\\kramaśaḥ karmayogī ca pañcaite muktibhājanāḥ\\\  \ SU.1.37\\karmayogasya yan mūlaṃ tad vakṣyāmi samāsataḥ\\liṅgam āyatanaṃ ceti tatra karma pravartate\\\ 

\end{document}

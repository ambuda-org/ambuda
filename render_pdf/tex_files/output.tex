\documentclass{tufte-handout}

\title{Kumārasaṃbhava}

\author{apauruṣeya}

%\date{28 March 2010} % without \date command, current date is supplied

%\geometry{showframe} % display margins for debugging page layout

\usepackage{graphicx} % allow embedded images
  \setkeys{Gin}{width=\linewidth,totalheight=\textheight,keepaspectratio}
  \graphicspath{{graphics/}} % set of paths to search for images
\usepackage{amsmath}  % extended mathematics
\usepackage{booktabs} % book-quality tables
\usepackage{units}    % non-stacked fractions and better unit spacing
\usepackage{multicol} % multiple column layout facilities
\usepackage{lipsum}   % filler text
\usepackage{fancyvrb} % extended verbatim environments
  \fvset{fontsize=\normalsize}% default font size for fancy-verbatim environments

% Standardize command font styles and environments
\newcommand{\doccmd}[1]{\texttt{\textbackslash#1}}% command name -- adds backslash automatically
\newcommand{\docopt}[1]{\ensuremath{\langle}\textrm{\textit{#1}}\ensuremath{\rangle}}% optional command argument
\newcommand{\docarg}[1]{\textrm{\textit{#1}}}% (required) command argument
\newcommand{\docenv}[1]{\textsf{#1}}% environment name
\newcommand{\docpkg}[1]{\texttt{#1}}% package name
\newcommand{\doccls}[1]{\texttt{#1}}% document class name
\newcommand{\docclsopt}[1]{\texttt{#1}}% document class option name
\newenvironment{docspec}{\begin{quote}\noindent}{\end{quote}}% command specification environment

\usepackage[parfill]{parskip}
\usepackage{fontspec}
\usepackage[english]{babel}
\usepackage{microtype}

% Sanskrit Support %
\babelprovide[import]{sanskrit-devanagari}
\babelprovide[import]{english}
\babelprovide[import]{greek}
\babelprovide[import]{german}
\babelfont[english]{rm}{Bitter}
\babelfont[sanskrit-devanagari]{rm}
          {Siddhanta}
\babelfont[sanskrit-devanagari]{sf}
          [Language=Default]{Noto Sans}

          
\newcommand\textsanskrit[1]{\foreignlanguage{sanskrit-devanagari}{#1}}
\newenvironment{sanskrit}%
{\begin{otherlanguage}{sanskrit-devanagari}}%
{\end{otherlanguage}}
%%%%%%%%%%%%%%%%%%


\begin{document}
\maketitle


    
	\begin{sanskrit}
	
	    
		Ks.1.1 
    
	    
		asty uttarasyāṃ diśi devatātmā 
    
	    
		himālayo nāma nagādhirājaḥ 
    
	\end{sanskrit}

    
	\begin{sanskrit}
	
	    
		Ks.1.1 
    
	    
		pūrvāparau toyanidhī vigāhya 
    
	    
		sthitaḥ pṛthivyā iva mānadaṇḍaḥ 
    
	\end{sanskrit}

    
	\begin{sanskrit}
	
	    
		Ks.1.2 
    
	    
		yaṃ sarvaśailāḥ parikalpya vatsaṃ 
    
	    
		merau sthite dogdhari dohadakṣe 
    
	\end{sanskrit}

    
	\begin{sanskrit}
	
	    
		Ks.1.2 
    
	    
		bhāsvanti ratnāni mahauṣadhīś ca 
    
	    
		pṛthūpadiṣṭāṃ duduhur dharitrīm 
    
	\end{sanskrit}

    
	\begin{sanskrit}
	
	    
		Ks.1.3 
    
	    
		anantaratnaprabhavasya yasya 
    
	    
		himaṃ na saubhāgyavilopi jātam 
    
	\end{sanskrit}

    
	\begin{sanskrit}
	
	    
		Ks.1.3 
    
	    
		eko hi doṣo guṇasaṃnipāte 
    
	    
		nimajjatīndoḥ kiraṇeṣv ivāṅkaḥ 
    
	\end{sanskrit}

    
	\begin{sanskrit}
	
	    
		Ks.1.4 
    
	    
		yaś cāpsarovibhramamaṇḍanānāṃ 
    
	    
		saṃpādayitrīṃ śikharair bibharti 
    
	\end{sanskrit}

    
	\begin{sanskrit}
	
	    
		Ks.1.4 
    
	    
		balāhakacchedavibhaktarāgām 
    
	    
		akālasaṃdhyām iva dhātumattām 
    
	\end{sanskrit}

    
	\begin{sanskrit}
	
	    
		Ks.1.5 
    
	    
		āmekhalaṃ saṃcaratāṃ ghanānāṃ 
    
	    
		cchāyām adhaḥsānugatāṃ niṣevya 
    
	\end{sanskrit}

    
	\begin{sanskrit}
	
	    
		Ks.1.5 
    
	    
		udvejitā vṛṣṭibhir āśrayante 
    
	    
		śṛṅgāṇi yasyātapavanti siddhāḥ 
    
	\end{sanskrit}

    
	\begin{sanskrit}
	
	    
		Ks.1.6 
    
	    
		padaṃ tuṣārasrutidhautaraktaṃ 
    
	    
		yasminn adṛṣṭvāpi hatadvipānām 
    
	\end{sanskrit}

    
	\begin{sanskrit}
	
	    
		Ks.1.6 
    
	    
		vidanti mārgaṃ nakharandhramuktair 
    
	    
		muktāphalaiḥ kesariṇāṃ kirātāḥ 
    
	\end{sanskrit}

    
	\begin{sanskrit}
	
	    
		Ks.1.7 
    
	    
		nyastākṣarā dhāturasena yatra 
    
	    
		bhūrjatvacaḥ kuñjarabinduśoṇāḥ 
    
	\end{sanskrit}

    
	\begin{sanskrit}
	
	    
		Ks.1.7 
    
	    
		vrajanti vidyādharasundarīṇām 
    
	    
		anaṅgalekhakriyayopayogam 
    
	\end{sanskrit}

    
	\begin{sanskrit}
	
	    
		Ks.1.8 
    
	    
		yaḥ pūrayan kīcakarandhrabhāgān 
    
	    
		darīmukhotthena samīraṇena 
    
	\end{sanskrit}

    
	\begin{sanskrit}
	
	    
		Ks.1.8 
    
	    
		udgāsyatām icchati kiṃnarāṇāṃ 
    
	    
		tānapradāyitvam ivopagantum 
    
	\end{sanskrit}

    
	\begin{sanskrit}
	
	    
		Ks.1.9 
    
	    
		kapolakaṇḍūḥ karibhir vinetuṃ 
    
	    
		vighaṭṭitānāṃ saraladrumāṇām 
    
	\end{sanskrit}

    
	\begin{sanskrit}
	
	    
		Ks.1.9 
    
	    
		yatra srutakṣīratayā prasūtaḥ 
    
	    
		sānūni gandhaḥ surabhīkaroti 
    
	\end{sanskrit}

    
	\begin{sanskrit}
	
	    
		Ks.1.10 
    
	    
		vanecarāṇāṃ vanitāsakhānāṃ 
    
	    
		darīgṛhotsaṅganiṣaktabhāsaḥ 
    
	\end{sanskrit}

    
	\begin{sanskrit}
	
	    
		Ks.1.10 
    
	    
		bhavanti yatrauṣadhayo rajanyām 
    
	    
		atailapūrāḥ suratapradīpāḥ 
    
	\end{sanskrit}

    
	\begin{sanskrit}
	
	    
		Ks.1.11 
    
	    
		udvejayaty aṅgulipārṣṇibhāgān 
    
	    
		mārge śilībhūtahime 'pi yatra 
    
	\end{sanskrit}

    
	\begin{sanskrit}
	
	    
		Ks.1.11 
    
	    
		na durvahaśroṇipayodharārtā 
    
	    
		bhindanti mandāṃ gatim aśvamukhyaḥ 
    
	\end{sanskrit}

    
	\begin{sanskrit}
	
	    
		Ks.1.12 
    
	    
		divākarād rakṣati yo guhāsu 
    
	    
		līnaṃ divā bhītam ivāndhakāram 
    
	\end{sanskrit}

    
	\begin{sanskrit}
	
	    
		Ks.1.12 
    
	    
		kṣudre 'pi nūnaṃ śaraṇaṃ prapanne 
    
	    
		mamatvam uccaiḥśirasāṃ satīva 
    
	\end{sanskrit}

    
	\begin{sanskrit}
	
	    
		Ks.1.13 
    
	    
		lāṅgūlavikṣepavisarpiśobhair 
    
	    
		itas tataś candramarīcigauraiḥ 
    
	\end{sanskrit}

    
	\begin{sanskrit}
	
	    
		Ks.1.13 
    
	    
		yasyārthayuktaṃ girirājaśabdaṃ 
    
	    
		kurvanti vālavyajanaiś camaryaḥ 
    
	\end{sanskrit}

    
	\begin{sanskrit}
	
	    
		Ks.1.14 
    
	    
		yatrāṃśukākṣepavilajjitānāṃ 
    
	    
		yadṛcchayā kiṃpuruṣāṅganānām 
    
	\end{sanskrit}

    
	\begin{sanskrit}
	
	    
		Ks.1.14 
    
	    
		darīgṛhadvāravilambibimbās 
    
	    
		tiraskariṇyo jaladā bhavanti 
    
	\end{sanskrit}

    
	\begin{sanskrit}
	
	    
		Ks.1.15 
    
	    
		bhāgīrathīnirjharasīkarāṇāṃ 
    
	    
		voḍhā muhuḥ kampitadevadāruḥ 
    
	\end{sanskrit}

    
	\begin{sanskrit}
	
	    
		Ks.1.15 
    
	    
		yad vāyur anviṣṭamṛgaiḥ kirātair 
    
	    
		āsevyate bhinnaśikhaṇḍibarhaḥ 
    
	\end{sanskrit}

    
	\begin{sanskrit}
	
	    
		Ks.1.16 
    
	    
		saptarṣihastāvacitāvaśeṣāṇy 
    
	    
		adho vivasvān parivartamānaḥ 
    
	\end{sanskrit}

    
	\begin{sanskrit}
	
	    
		Ks.1.16 
    
	    
		padmāni yasyāgrasaroruhāṇi 
    
	    
		prabodhayaty ūrdhvamukhair mayūkhaiḥ 
    
	\end{sanskrit}

    
	\begin{sanskrit}
	
	    
		Ks.1.17 
    
	    
		yajñāṅgayonitvam avekṣya yasya 
    
	    
		sāraṃ dharitrīdharaṇakṣamaṃ ca 
    
	\end{sanskrit}

    
	\begin{sanskrit}
	
	    
		Ks.1.17 
    
	    
		prajāpatiḥ kalpitayajñabhāgaṃ 
    
	    
		śailādhipatyaṃ svayam anvatiṣṭhat 
    
	\end{sanskrit}

    
	\begin{sanskrit}
	
	    
		Ks.1.18 
    
	    
		sa mānasīṃ merusakhaḥ pitṝṇāṃ 
    
	    
		kanyāṃ kulasya sthitaye sthitijñaḥ 
    
	\end{sanskrit}

    
	\begin{sanskrit}
	
	    
		Ks.1.18 
    
	    
		menāṃ munīnām api mānanīyām 
    
	    
		ātmānurūpāṃ vidhinopayeme 
    
	\end{sanskrit}

    
	\begin{sanskrit}
	
	    
		Ks.1.19 
    
	    
		kālakrameṇātha tayoḥ pravṛtte 
    
	    
		svarūpayogye surataprasaṅge 
    
	\end{sanskrit}

    
	\begin{sanskrit}
	
	    
		Ks.1.19 
    
	    
		manoramaṃ yauvanam udvahantyā 
    
	    
		garbho 'bhavad bhūdhararājapatnyāḥ 
    
	\end{sanskrit}

    
	\begin{sanskrit}
	
	    
		Ks.1.20 
    
	    
		asūta sā nāgavadhūpabhogyaṃ 
    
	    
		mainākam ambhonidhibaddhasakhyam 
    
	\end{sanskrit}

    
	\begin{sanskrit}
	
	    
		Ks.1.20 
    
	    
		kruddhe 'pi pakṣacchidi vṛtraśatrāv 
    
	    
		avedanājñaṃ kuliśakṣatānām 
    
	\end{sanskrit}

    
	\begin{sanskrit}
	
	    
		Ks.1.21 
    
	    
		athāvamānena pituḥ prayuktā 
    
	    
		dakṣasya kanyā bhavapūrvapatnī 
    
	\end{sanskrit}

    
	\begin{sanskrit}
	
	    
		Ks.1.21 
    
	    
		satī satī yogavisṛṣṭadehā 
    
	    
		tāṃ janmane śailavadhūṃ prapede 
    
	\end{sanskrit}

    
	\begin{sanskrit}
	
	    
		Ks.1.22 
    
	    
		sā bhūdharāṇām adhipena tasyāṃ 
    
	    
		samādhimatyām udapādi bhavyā 
    
	\end{sanskrit}

    
	\begin{sanskrit}
	
	    
		Ks.1.22 
    
	    
		samyakprayogād aparikṣatāyāṃ 
    
	    
		nītāv ivotsāhaguṇena saṃpat 
    
	\end{sanskrit}

    
	\begin{sanskrit}
	
	    
		Ks.1.23 
    
	    
		prasannadik pāṃsuviviktavātaṃ 
    
	    
		śaṅkhasvanānantarapuṣpavṛṣṭi 
    
	\end{sanskrit}

    
	\begin{sanskrit}
	
	    
		Ks.1.23 
    
	    
		śarīriṇāṃ sthāvarajaṅgamānāṃ 
    
	    
		sukhāya tajjanmadinaṃ babhūva 
    
	\end{sanskrit}

    
	\begin{sanskrit}
	
	    
		Ks.1.24 
    
	    
		tayā duhitrā sutarāṃ savitrī 
    
	    
		sphuratprabhāmaṇḍalayā cakāse 
    
	\end{sanskrit}

    
	\begin{sanskrit}
	
	    
		Ks.1.24 
    
	    
		vidūrabhūmir navameghaśabdād 
    
	    
		udbhinnayā ratnaśalākayeva 
    
	\end{sanskrit}

    
	\begin{sanskrit}
	
	    
		Ks.1.25 
    
	    
		dine dine sā parivardhamānā 
    
	    
		labdhodayā cāndramasīva lekhā 
    
	\end{sanskrit}

    
	\begin{sanskrit}
	
	    
		Ks.1.25 
    
	    
		pupoṣa lāvaṇyamayān viśeṣāñ 
    
	    
		jyotsnāntarāṇīva kalāntarāṇi 
    
	\end{sanskrit}

    
	\begin{sanskrit}
	
	    
		Ks.1.26 
    
	    
		tāṃ pārvatīty ābhijanena nāmnā 
    
	    
		bandhupriyāṃ bandhujano juhāva 
    
	\end{sanskrit}

    
	\begin{sanskrit}
	
	    
		Ks.1.26 
    
	    
		u meti mātrā tapaso niṣiddhā 
    
	    
		paścād umākhyāṃ sumukhī jagāma 
    
	\end{sanskrit}

    
	\begin{sanskrit}
	
	    
		Ks.1.27 
    
	    
		mahībhṛtaḥ putravato 'pi dṛṣṭis 
    
	    
		tasminn apatye na jagāma tṛptim 
    
	\end{sanskrit}

    
	\begin{sanskrit}
	
	    
		Ks.1.27 
    
	    
		anantapuṣpasya madhor hi cūte 
    
	    
		dvirephamālā saviśeṣasaṅgā 
    
	\end{sanskrit}

    
	\begin{sanskrit}
	
	    
		Ks.1.28 
    
	    
		prabhāmahatyā śikhayeva dīpas 
    
	    
		trimārgayeva tridivasya mārgaḥ 
    
	\end{sanskrit}

    
	\begin{sanskrit}
	
	    
		Ks.1.28 
    
	    
		saṃskāravatyeva girā manīṣī 
    
	    
		tayā sa pūtaś ca vibhūṣitaś ca 
    
	\end{sanskrit}

    
	\begin{sanskrit}
	
	    
		Ks.1.29 
    
	    
		mandākinīsaikatavedikābhiḥ 
    
	    
		sā kandukaiḥ kṛtrimaputrakaiś ca 
    
	\end{sanskrit}

    
	\begin{sanskrit}
	
	    
		Ks.1.29 
    
	    
		reme muhur madhyagatā sakhīnāṃ 
    
	    
		krīḍārasaṃ nirviśatīva bālye 
    
	\end{sanskrit}

    
	\begin{sanskrit}
	
	    
		Ks.1.30 
    
	    
		tāṃ haṃsamālāḥ śaradīva gaṅgāṃ 
    
	    
		mahauṣadhiṃ naktam ivātmabhāsaḥ 
    
	\end{sanskrit}

    
	\begin{sanskrit}
	
	    
		Ks.1.30 
    
	    
		sthiropadeśām upadeśakāle 
    
	    
		prapedire prāktanajanmavidyāḥ 
    
	\end{sanskrit}

    
	\begin{sanskrit}
	
	    
		Ks.1.31 
    
	    
		asaṃbhṛtaṃ maṇḍanam aṅgayaṣṭer 
    
	    
		anāsavākhyaṃ karaṇaṃ madasya 
    
	\end{sanskrit}

    
	\begin{sanskrit}
	
	    
		Ks.1.31 
    
	    
		kāmasya puṣpavyatiriktam astraṃ 
    
	    
		bālyāt paraṃ sātha vayaḥ prapede 
    
	\end{sanskrit}

    
	\begin{sanskrit}
	
	    
		Ks.1.32 
    
	    
		unmīlitaṃ tūlikayeva citraṃ 
    
	    
		sūryāṃśubhir bhinnam ivāravindam 
    
	\end{sanskrit}

    
	\begin{sanskrit}
	
	    
		Ks.1.32 
    
	    
		babhūva tasyāś caturasraśobhi 
    
	    
		vapur vibhaktaṃ navayauvanena 
    
	\end{sanskrit}

    
	\begin{sanskrit}
	
	    
		Ks.1.33 
    
	    
		abhyunnatāṅguṣṭhanakhaprabhābhir 
    
	    
		nikṣepaṇād rāgam ivodgirantau 
    
	\end{sanskrit}

    
	\begin{sanskrit}
	
	    
		Ks.1.33 
    
	    
		ājahratus taccaraṇau pṛthivyāṃ 
    
	    
		sthalāravindaśriyam avyavasthām 
    
	\end{sanskrit}

    
	\begin{sanskrit}
	
	    
		Ks.1.34 
    
	    
		sā rājahaṃsair iva saṃnatāṅgī 
    
	    
		gateṣu līlāñcitavikrameṣu 
    
	\end{sanskrit}

    
	\begin{sanskrit}
	
	    
		Ks.1.34 
    
	    
		vyanīyata pratyupadeśalubdhair 
    
	    
		āditsubhir nūpurasiñjitāni 
    
	\end{sanskrit}

    
	\begin{sanskrit}
	
	    
		Ks.1.35 
    
	    
		vṛttānupūrve ca na cātidīrghe 
    
	    
		jaṅghe śubhe sṛṣṭavatas tadīye 
    
	\end{sanskrit}

    
	\begin{sanskrit}
	
	    
		Ks.1.35 
    
	    
		śeṣāṅganirmāṇavidhau vidhātur 
    
	    
		lāvaṇya utpādya ivāsa yatnaḥ 
    
	\end{sanskrit}

    
	\begin{sanskrit}
	
	    
		Ks.1.36 
    
	    
		nāgendrahastās tvaci karkaśatvād 
    
	    
		ekāntaśaityāt kadalīviśeṣāḥ 
    
	\end{sanskrit}

    
	\begin{sanskrit}
	
	    
		Ks.1.36 
    
	    
		labdhvāpi loke pariṇāhi rūpaṃ 
    
	    
		jātās tadūrvor upamānabāhyāḥ 
    
	\end{sanskrit}

    
	\begin{sanskrit}
	
	    
		Ks.1.37 
    
	    
		etāvatā nanv anumeyaśobhaṃ 
    
	    
		kāñcīguṇasthānam aninditāyāḥ 
    
	\end{sanskrit}

    
	\begin{sanskrit}
	
	    
		Ks.1.37 
    
	    
		āropitaṃ yad giriśena paścād 
    
	    
		ananyanārīkamanīyam aṅkam 
    
	\end{sanskrit}

    
	\begin{sanskrit}
	
	    
		Ks.1.38 
    
	    
		tasyāḥ praviṣṭā natanābhirandhraṃ 
    
	    
		rarāja tanvī navalomarājiḥ 
    
	\end{sanskrit}

    
	\begin{sanskrit}
	
	    
		Ks.1.38 
    
	    
		nīvīm atikramya sitetarasya 
    
	    
		tanmekhalāmadhyamaṇer ivārciḥ 
    
	\end{sanskrit}

    
	\begin{sanskrit}
	
	    
		Ks.1.39 
    
	    
		madhyena sā vedivilagnamadhyā 
    
	    
		valitrayaṃ cāru babhāra bālā 
    
	\end{sanskrit}

    
	\begin{sanskrit}
	
	    
		Ks.1.39 
    
	    
		ārohaṇārthaṃ navayauvanena 
    
	    
		kāmasya sopānam iva prayuktam 
    
	\end{sanskrit}

    
	\begin{sanskrit}
	
	    
		Ks.1.40 
    
	    
		anyonyam utpīḍayad utpalākṣyāḥ 
    
	    
		stanadvayaṃ pāṇḍu tathā pravṛddham 
    
	\end{sanskrit}

    
	\begin{sanskrit}
	
	    
		Ks.1.40 
    
	    
		madhye yathā śyāmamukhasya tasya 
    
	    
		mṛṇālasūtrāntaram apy alabhyam 
    
	\end{sanskrit}

    
	\begin{sanskrit}
	
	    
		Ks.1.41 
    
	    
		śirīṣapuṣpādhikasaukumāryau 
    
	    
		bāhū tadīyāv iti me vitarkaḥ 
    
	\end{sanskrit}

    
	\begin{sanskrit}
	
	    
		Ks.1.41 
    
	    
		parājitenāpi kṛtau harasya 
    
	    
		yau kaṇṭhapāśau makaradhvajena 
    
	\end{sanskrit}

    
	\begin{sanskrit}
	
	    
		Ks.1.42 
    
	    
		kaṇṭhasya tasyāḥ stanabandhurasya 
    
	    
		muktākalāpasya ca nistalasya 
    
	\end{sanskrit}

    
	\begin{sanskrit}
	
	    
		Ks.1.42 
    
	    
		anyonyaśobhājananād babhūva 
    
	    
		sādhāraṇo bhūṣaṇabhūṣyabhāvaḥ 
    
	\end{sanskrit}

    
	\begin{sanskrit}
	
	    
		Ks.1.43 
    
	    
		candraṃ gatā padmaguṇān na bhuṅkte 
    
	    
		padmāśritā cāndramasīm abhikhyām 
    
	\end{sanskrit}

    
	\begin{sanskrit}
	
	    
		Ks.1.43 
    
	    
		umāmukhaṃ tu pratipadya lolā 
    
	    
		dvisaṃśrayāṃ prītim avāpa lakṣmīḥ 
    
	\end{sanskrit}

    
	\begin{sanskrit}
	
	    
		Ks.1.44 
    
	    
		puṣpaṃ pravālopahitaṃ yadi syān 
    
	    
		muktāphalaṃ vā sphuṭavidrumastham 
    
	\end{sanskrit}

    
	\begin{sanskrit}
	
	    
		Ks.1.44 
    
	    
		tato 'nukuryād viśadasya tasyās 
    
	    
		tāmrauṣṭhaparyastarucaḥ smitasya 
    
	\end{sanskrit}

    
	\begin{sanskrit}
	
	    
		Ks.1.45 
    
	    
		svareṇa tasyām amṛtasruteva 
    
	    
		prajalpitāyām abhijātavāci 
    
	\end{sanskrit}

    
	\begin{sanskrit}
	
	    
		Ks.1.45 
    
	    
		apy anyapuṣṭā pratikūlaśabdā 
    
	    
		śrotur vitantrīr iva tāḍyamānā 
    
	\end{sanskrit}

    
	\begin{sanskrit}
	
	    
		Ks.1.46 
    
	    
		pravātanīlotpalanirviśeṣam 
    
	    
		adhīraviprekṣitam āyatākṣyā 
    
	\end{sanskrit}

    
	\begin{sanskrit}
	
	    
		Ks.1.46 
    
	    
		tayā gṛhītaṃ nu mṛgāṅganābhyas 
    
	    
		tato gṛhītaṃ nu mṛgāṅganābhiḥ 
    
	\end{sanskrit}

    
	\begin{sanskrit}
	
	    
		Ks.1.47 
    
	    
		tasyāḥ śalākāñjananirmiteva 
    
	    
		kāntir bhruvor ānatalekhayor yā 
    
	\end{sanskrit}

    
	\begin{sanskrit}
	
	    
		Ks.1.47 
    
	    
		tāṃ vīkṣya līlācaturām anaṅgaḥ 
    
	    
		svacāpasaundaryamadaṃ mumoca 
    
	\end{sanskrit}

    
	\begin{sanskrit}
	
	    
		Ks.1.48 
    
	    
		lajjā tiraścāṃ yadi cetasi syād 
    
	    
		asaṃśayaṃ parvatarājaputryāḥ 
    
	\end{sanskrit}

    
	\begin{sanskrit}
	
	    
		Ks.1.48 
    
	    
		taṃ keśapāśaṃ prasamīkṣya kuryur 
    
	    
		vālapriyatvaṃ śithilaṃ camaryaḥ 
    
	\end{sanskrit}

    
	\begin{sanskrit}
	
	    
		Ks.1.49 
    
	    
		sarvopamādravyasamuccayena 
    
	    
		yathāpradeśaṃ viniveśitena 
    
	\end{sanskrit}

    
	\begin{sanskrit}
	
	    
		Ks.1.49 
    
	    
		sā nirmitā viśvasṛjā prayatnād 
    
	    
		ekasthasaundaryadidṛkṣayeva 
    
	\end{sanskrit}

    
	\begin{sanskrit}
	
	    
		Ks.1.50 
    
	    
		tāṃ nāradaḥ kāmacaraḥ kadā cit 
    
	    
		kanyāṃ kila prekṣya pituḥ samīpe 
    
	\end{sanskrit}

    
	\begin{sanskrit}
	
	    
		Ks.1.50 
    
	    
		samādideśaikavadhūṃ bhavitrīṃ 
    
	    
		premṇā śarīrārdhaharāṃ harasya 
    
	\end{sanskrit}

    
	\begin{sanskrit}
	
	    
		Ks.1.51 
    
	    
		guruḥ pragalbhe 'pi vayasy ato 'syās 
    
	    
		tasthau nivṛttānyavarābhilāṣaḥ 
    
	\end{sanskrit}

    
	\begin{sanskrit}
	
	    
		Ks.1.51 
    
	    
		ṛte kṛśānor na hi mantrapūtam 
    
	    
		arhanti tejāṃsy aparāṇi havyam 
    
	\end{sanskrit}

    
	\begin{sanskrit}
	
	    
		Ks.1.52 
    
	    
		ayācitāraṃ na hi devadevam 
    
	    
		adriḥ sutāṃ grāhayituṃ śaśāka 
    
	\end{sanskrit}

    
	\begin{sanskrit}
	
	    
		Ks.1.52 
    
	    
		abhyarthanābhaṅgabhayena sādhur 
    
	    
		mādhyasthyam iṣṭe 'py avalambate 'rthe 
    
	\end{sanskrit}

    
	\begin{sanskrit}
	
	    
		Ks.1.53 
    
	    
		yadaiva pūrve janane śarīraṃ 
    
	    
		sā dakṣaroṣāt sudatī sasarja 
    
	\end{sanskrit}

    
	\begin{sanskrit}
	
	    
		Ks.1.53 
    
	    
		tadāprabhṛty eva vimuktasaṅgaḥ 
    
	    
		patiḥ paśūnām aparigraho 'bhūt 
    
	\end{sanskrit}

    
	\begin{sanskrit}
	
	    
		Ks.1.54 
    
	    
		sa kṛttivāsās tapase yatātmā 
    
	    
		gaṅgāpravāhokṣitadevadāru 
    
	\end{sanskrit}

    
	\begin{sanskrit}
	
	    
		Ks.1.54 
    
	    
		prasthaṃ himādrer mṛganābhigandhi 
    
	    
		kiṃ cit kvaṇatkiṃnaram adhyuvāsa 
    
	\end{sanskrit}

    
	\begin{sanskrit}
	
	    
		Ks.1.55 
    
	    
		gaṇā nameruprasavāvataṃsā 
    
	    
		bhūrjatvacaḥ sparśavatīr dadhānāḥ 
    
	\end{sanskrit}

    
	\begin{sanskrit}
	
	    
		Ks.1.55 
    
	    
		manaḥśilāvicchuritā niṣeduḥ 
    
	    
		śaileyanaddheṣu śilātaleṣu 
    
	\end{sanskrit}

    
	\begin{sanskrit}
	
	    
		Ks.1.56 
    
	    
		tuṣārasaṃghātaśilāḥ khurāgraiḥ 
    
	    
		samullikhan darpakalaḥ kakudmān 
    
	\end{sanskrit}

    
	\begin{sanskrit}
	
	    
		Ks.1.56 
    
	    
		dṛṣṭaḥ kathaṃ cid gavayair vivignair 
    
	    
		asoḍhasiṃhadhvanir unnanāda 
    
	\end{sanskrit}

    
	\begin{sanskrit}
	
	    
		Ks.1.57 
    
	    
		tatrāgnim ādhāya samitsamiddhaṃ 
    
	    
		svam eva mūrtyantaram aṣṭamūrtiḥ 
    
	\end{sanskrit}

    
	\begin{sanskrit}
	
	    
		Ks.1.57 
    
	    
		svayaṃ vidhātā tapasaḥ phalānām 
    
	    
		kenāpi kāmena tapaś cacāra 
    
	\end{sanskrit}

    
	\begin{sanskrit}
	
	    
		Ks.1.58 
    
	    
		anarghyam arghyeṇa tam adrināthaḥ 
    
	    
		svargaukasām arcitam arcayitvā 
    
	\end{sanskrit}

    
	\begin{sanskrit}
	
	    
		Ks.1.58 
    
	    
		ārādhanāyāsya sakhīsametāṃ 
    
	    
		samādideśa prayatāṃ tanūjām 
    
	\end{sanskrit}

    
	\begin{sanskrit}
	
	    
		Ks.1.59 
    
	    
		pratyarthibhūtām api tāṃ samādheḥ 
    
	    
		śuśrūṣamāṇāṃ giriśo 'numene 
    
	\end{sanskrit}

    
	\begin{sanskrit}
	
	    
		Ks.1.59 
    
	    
		vikārahetau sati vikriyante 
    
	    
		yeṣāṃ na cetāṃsi ta eva dhīrāḥ 
    
	\end{sanskrit}

    
	\begin{sanskrit}
	
	    
		Ks.1.60 
    
	    
		avacitabalipuṣpā vedisaṃmārgadakṣā 
    
	    
		niyamavidhijalānāṃ barhiṣāṃ copanetrī 
    
	\end{sanskrit}

    
	\begin{sanskrit}
	
	    
		Ks.1.60 
    
	    
		giriśam upacacāra pratyahaṃ sā sukeśī 
    
	    
		niyamitaparikhedā tacchiraścandrapādaiḥ 
    
	\end{sanskrit}



\end{document}
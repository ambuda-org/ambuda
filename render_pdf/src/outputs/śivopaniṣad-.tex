\documentclass{article}

\title{śivopaniṣad}

\author{}

%\date{28 March 2010} % without \date command, current date is supplied


\usepackage[parfill]{parskip}
\usepackage{fontspec}

\begin{document}
\maketitle


    
	%\begin{}
	
	    
		SU.1.1 
    
	    
		kailāsaśikharāsīnam 
    
	    
		aśeṣāmarapūjitam 
    
	%\end{}

    
	%\begin{}
	
	    
		SU.1.1 
    
	    
		kālaghnaṃ śrīmahākālam 
    
	    
		īśvaraṃ jñānapāragam 
    
	%\end{}

    
	%\begin{}
	
	    
		None 
    
	    
		karma-yogasya yan mūlaṃ 
    
	    
		tad vakṣyāmi samāsataḥ 
    
	%\end{}

    
	%\begin{}
	
	    
		None 
    
	    
		liṅgam āyatanaṃ ca iti 
    
	    
		tatra karma pravartate 
    
	%\end{}

    
	%\begin{}
	
	    
		SU.1.2 
    
	    
		saṃpūjya vidhivad bhaktyā 
    
	    
		ṛṣyātreyaḥ susaṃyataḥ 
    
	%\end{}

    
	%\begin{}
	
	    
		SU.1.2 
    
	    
		sarvabhūtahitārthāya 
    
	    
		papracchedaṃ mahāmuniḥ 
    
	%\end{}

    
	%\begin{}
	
	    
		SU.1.3 
    
	    
		jñānayogaṃ na vindanti 
    
	    
		ye narā mandabuddhayaḥ 
    
	%\end{}

    
	%\begin{}
	
	    
		SU.1.3 
    
	    
		te mucyante kathaṃ ghorād 
    
	    
		bhagavan bhavasāgarāt 
    
	%\end{}

    
	%\begin{}
	
	    
		SU.1.4 
    
	    
		evaṃ pṛṣṭaḥ prasannātmā 
    
	    
		ṛṣy ātreyeṇa dhīmatā 
    
	%\end{}

    
	%\begin{}
	
	    
		SU.1.4 
    
	    
		mandabuddhivimuktyarthaṃ 
    
	    
		mahākālaḥ prabhāṣate 
    
	%\end{}

    
	%\begin{}
	
	    
		SU.1.5 
    
	    
		purā rudreṇa gaditāḥ 
    
	    
		śivadharmāḥ sanātanāḥ 
    
	%\end{}

    
	%\begin{}
	
	    
		SU.1.5 
    
	    
		devyāḥ sarvagaṇānāṃ ca 
    
	    
		saṃkṣepād granthakoṭibhiḥ 
    
	%\end{}

    
	%\begin{}
	
	    
		SU.1.6 
    
	    
		āyuḥ prajñāṃ tathā śaktiṃ 
    
	    
		prasamīkṣya nṝṇām iha 
    
	%\end{}

    
	%\begin{}
	
	    
		SU.1.6 
    
	    
		tāpatrayaprapīḍāṃ ca 
    
	    
		bhogatṛṣṇāvimohinīm 
    
	%\end{}

    
	%\begin{}
	
	    
		SU.1.7 
    
	    
		te dharmāḥ skandanandibhyām 
    
	    
		anyaiś ca munisattamaiḥ 
    
	%\end{}

    
	%\begin{}
	
	    
		SU.1.7 
    
	    
		sāramādāya nirdiṣṭāḥ 
    
	    
		samyakprakaraṇāntaraiḥ 
    
	%\end{}

    
	%\begin{}
	
	    
		SU.1.8 
    
	    
		sārād api mahāsāraṃ 
    
	    
		śivopaniṣadaṃ param 
    
	%\end{}

    
	%\begin{}
	
	    
		SU.1.8 
    
	    
		alpagranthaṃ mahārthaṃ ca 
    
	    
		pravakṣyāmi jagaddhitam 
    
	%\end{}

    
	%\begin{}
	
	    
		SU.1.9 
    
	    
		śivaḥ śiva ime śānta 
    
	    
		nāma cādyaṃ muhurmuhuḥ 
    
	%\end{}

    
	%\begin{}
	
	    
		SU.1.9 
    
	    
		uccārayanti tad bhaktyā 
    
	    
		te śivā nātra saṃśayaḥ 
    
	%\end{}

    
	%\begin{}
	
	    
		SU.1.10 
    
	    
		aśivāḥ pāśasaṃyuktāḥ 
    
	    
		paśavaḥ sarvacetanāḥ 
    
	%\end{}

    
	%\begin{}
	
	    
		SU.1.10 
    
	    
		yasmād vilakṣaṇās tebhyas 
    
	    
		tasmād īśaḥ śivaḥ smṛtaḥ 
    
	%\end{}

    
	%\begin{}
	
	    
		SU.1.11 
    
	    
		guṇo buddhir ahaṃkāras 
    
	    
		tanmātrāṇīndriyāni ca 
    
	%\end{}

    
	%\begin{}
	
	    
		SU.1.11 
    
	    
		bhūtāni ca caturviṃśad 
    
	    
		iti pāśāḥ prakīrtitāḥ 
    
	%\end{}

    
	%\begin{}
	
	    
		SU.1.12 
    
	    
		pañcaviṃśakam ajñānaṃ 
    
	    
		sahajaṃ sarvadehinām 
    
	%\end{}

    
	%\begin{}
	
	    
		SU.1.12 
    
	    
		pāśājālasya tan mūlaṃ 
    
	    
		prakṛtiḥ kāraṇāya naḥ 
    
	%\end{}

    
	%\begin{}
	
	    
		SU.1.13 
    
	    
		satyajñāne nibadhyante 
    
	    
		puruṣāḥ pāśabandhanaiḥ 
    
	%\end{}

    
	%\begin{}
	
	    
		SU.1.13 
    
	    
		madbhāvāc ca vimucyante 
    
	    
		jñāninaḥ pāśapañjarāt 
    
	%\end{}

    
	%\begin{}
	
	    
		SU.1.14 
    
	    
		ṣaḍviṃśakaś ca puruṣaḥ 
    
	    
		paśur ajñaḥ śivāgame 
    
	%\end{}

    
	%\begin{}
	
	    
		SU.1.14 
    
	    
		saptaviṃśa iti proktaḥ 
    
	    
		śivaḥ sarvajagatpatiḥ 
    
	%\end{}

    
	%\begin{}
	
	    
		SU.1.15 
    
	    
		yasmāc chivaḥ susaṃpūrṇaḥ 
    
	    
		sarvajñaḥ sarvagaḥ prabhuḥ 
    
	%\end{}

    
	%\begin{}
	
	    
		SU.1.15 
    
	    
		tasmāt sa pāśaharitaḥ 
    
	    
		sa viśuddhaḥ svabhāvataḥ 
    
	%\end{}

    
	%\begin{}
	
	    
		SU.1.16 
    
	    
		paśupāśaparaḥ śāntaḥ 
    
	    
		paramajñānadeśikaḥ 
    
	%\end{}

    
	%\begin{}
	
	    
		SU.1.16 
    
	    
		śivaḥ śivāya bhūtānāṃ 
    
	    
		taṃ vijñāya vimucyate 
    
	%\end{}

    
	%\begin{}
	
	    
		SU.1.17 
    
	    
		etad eva paraṃ jñānaṃ 
    
	    
		śiva ity akṣaradvayam 
    
	%\end{}

    
	%\begin{}
	
	    
		SU.1.17 
    
	    
		vicārād yāti vistāraṃ 
    
	    
		tailabindur ivāmbhasi 
    
	%\end{}

    
	%\begin{}
	
	    
		SU.1.18 
    
	    
		sakṛd uccāritaṃ yena 
    
	    
		śiva ity akṣaradvayam 
    
	%\end{}

    
	%\begin{}
	
	    
		SU.1.18 
    
	    
		baddhaḥ parikaras tena 
    
	    
		mokṣopagamanaṃ prati 
    
	%\end{}

    
	%\begin{}
	
	    
		SU.1.19 
    
	    
		dvyakṣaraḥ śivamantro 'yaṃ 
    
	    
		śivopaniṣadi smṛtaḥ 
    
	%\end{}

    
	%\begin{}
	
	    
		SU.1.19 
    
	    
		ekākṣaraḥ punaś cāyam 
    
	    
		om ity evaṃ vyavasthitaḥ 
    
	%\end{}

    
	%\begin{}
	
	    
		SU.1.20 
    
	    
		nāmasaṃkīrtaṇād eva 
    
	    
		śivasyāśeṣapātakaiḥ 
    
	%\end{}

    
	%\begin{}
	
	    
		SU.1.20 
    
	    
		yataḥ pramucyate kṣipraṃ 
    
	    
		mantro 'yaṃ dvyakṣaraḥ paraḥ 
    
	%\end{}

    
	%\begin{}
	
	    
		SU.1.21 
    
	    
		yaḥ śivaṃ śivam ity evaṃ 
    
	    
		dvyakṣaraṃ mantram abhyaset 
    
	%\end{}

    
	%\begin{}
	
	    
		SU.1.21 
    
	    
		ekākṣaraṃ vā satataṃ 
    
	    
		sa yāti paramaṃ padam 
    
	%\end{}

    
	%\begin{}
	
	    
		SU.1.22 
    
	    
		mitrasvajanabandhūnāṃ 
    
	    
		kuryān nāma śivātmakam 
    
	%\end{}

    
	%\begin{}
	
	    
		SU.1.22 
    
	    
		api tat kīrtanād yāti 
    
	    
		pāpamuktaḥ śivaṃ puram 
    
	%\end{}

    
	%\begin{}
	
	    
		SU.1.23 
    
	    
		vijñeyaḥ sa śivaḥ śānto 
    
	    
		naras tadbhāvabhāvitaḥ 
    
	%\end{}

    
	%\begin{}
	
	    
		SU.1.23 
    
	    
		āste sadā nirudvignaḥ 
    
	    
		sa dehānte vimucyate 
    
	%\end{}

    
	%\begin{}
	
	    
		SU.1.24 
    
	    
		hṛdy antaḥkaraṇaṃ jñeyaṃ 
    
	    
		śivasya āyatanaṃ param 
    
	%\end{}

    
	%\begin{}
	
	    
		SU.1.24 
    
	    
		hṛtpadmaṃ vedikā tatra 
    
	    
		liṅgam oṃkāram iṣyate 
    
	%\end{}

    
	%\begin{}
	
	    
		SU.1.25 
    
	    
		puruṣaḥ sthāpako jñeyaḥ 
    
	    
		satyaṃ saṃmārjanaṃ smṛtam 
    
	%\end{}

    
	%\begin{}
	
	    
		SU.1.25 
    
	    
		ahiṃsā gomayaṃ proktaṃ 
    
	    
		śāntiś ca salilaṃ param 
    
	%\end{}

    
	%\begin{}
	
	    
		SU.1.26 
    
	    
		kuryāt saṃmārjanaṃ prājño 
    
	    
		vairāgyaṃ candanaṃ smṛtam 
    
	%\end{}

    
	%\begin{}
	
	    
		SU.1.26 
    
	    
		pūjayed dhyānayogena 
    
	    
		saṃtoṣaiḥ kusumaiḥ sitaiḥ 
    
	%\end{}

    
	%\begin{}
	
	    
		SU.1.27 
    
	    
		dhūpaś ca guggulur deyaḥ 
    
	    
		prāṇāyāmasamudbhavaḥ 
    
	%\end{}

    
	%\begin{}
	
	    
		SU.1.27 
    
	    
		pratyāhāraś ca naivedyam 
    
	    
		asteyaṃ ca pradakṣiṇam 
    
	%\end{}

    
	%\begin{}
	
	    
		SU.1.28 
    
	    
		iti divyopacāraiś ca 
    
	    
		saṃpūjya paramaṃ śivam 
    
	%\end{}

    
	%\begin{}
	
	    
		SU.1.28 
    
	    
		japed dhyāyec ca muktyarthaṃ 
    
	    
		sarvasaṅgavivarjitaḥ 
    
	%\end{}

    
	%\begin{}
	
	    
		SU.1.29 
    
	    
		jñānayogavinirmuktaḥ 
    
	    
		karmayogasamāvṛttaḥ 
    
	%\end{}

    
	%\begin{}
	
	    
		SU.1.29 
    
	    
		mṛtaḥ śivapuraṃ gacchet 
    
	    
		sa tena śivakarmaṇā 
    
	%\end{}

    
	%\begin{}
	
	    
		SU.1.30 
    
	    
		tatra bhuktvā mahābhogān 
    
	    
		pralaye sarvadehinām 
    
	%\end{}

    
	%\begin{}
	
	    
		SU.1.30 
    
	    
		śivadharmāc chivajñānaṃ 
    
	    
		prāpya muktim avāpnuyāt 
    
	%\end{}

    
	%\begin{}
	
	    
		SU.1.31 
    
	    
		jñānayogena mucyante 
    
	    
		dehapātād anantaram 
    
	%\end{}

    
	%\begin{}
	
	    
		SU.1.31 
    
	    
		bhogān bhuktvā ca mucyante 
    
	    
		pralaye karmayoginaḥ 
    
	%\end{}

    
	%\begin{}
	
	    
		SU.1.32 
    
	    
		tasmāj jñānavido yogāt 
    
	    
		tathājñāḥ karmayoginaḥ 
    
	%\end{}

    
	%\begin{}
	
	    
		SU.1.32 
    
	    
		sarva eva vimucyante 
    
	    
		ye narāḥ śivam āśritāḥ 
    
	%\end{}

    
	%\begin{}
	
	    
		SU.1.33 
    
	    
		sa bhogaḥ śivavidyārthaṃ 
    
	    
		yeṣāṃ karmāsti nirmalam 
    
	%\end{}

    
	%\begin{}
	
	    
		SU.1.33 
    
	    
		te bhogān prāpya mucyante 
    
	    
		pralaye śivavidyayā 
    
	%\end{}

    
	%\begin{}
	
	    
		SU.1.34 
    
	    
		vidyā saṃkīrtanīyā hi 
    
	    
		yeṣāṃ karma na vidyate 
    
	%\end{}

    
	%\begin{}
	
	    
		SU.1.34 
    
	    
		te cāvartya vimucyante 
    
	    
		yāvat karma na tad bhavet 
    
	%\end{}

    
	%\begin{}
	
	    
		SU.1.35 
    
	    
		śivajñānavidaṃ tasmāt 
    
	    
		pūjayed vibhavair gurum 
    
	%\end{}

    
	%\begin{}
	
	    
		SU.1.35 
    
	    
		vidyādānaṃ ca kurvīta 
    
	    
		bhogamokṣajigīṣayā 
    
	%\end{}

    
	%\begin{}
	
	    
		SU.1.36 
    
	    
		śivayogī śivajñānī 
    
	    
		śivajāpī tapo'dhikaḥ 
    
	%\end{}

    
	%\begin{}
	
	    
		SU.1.36 
    
	    
		kramaśaḥ karmayogī ca 
    
	    
		pañcaite muktibhājanāḥ 
    
	%\end{}

    
	%\begin{}
	
	    
		SU.1.37 
    
	    
		karmayogasya yan mūlaṃ 
    
	    
		tad vakṣyāmi samāsataḥ 
    
	%\end{}

    
	%\begin{}
	
	    
		SU.1.37 
    
	    
		liṅgam āyatanaṃ ceti 
    
	    
		tatra karma pravartate 
    
	%\end{}


\end{document}